\renewcommand{\abstractnamefont}{\normalfont\Large\bfseries}
%\renewcommand{\abstracttextfont}{\normalfont\Huge}

\begin{abstract}
\hskip7mm

\begin{spacing}{1.3}

Janvier 2021 marque le début de la constitution des groupes pour les différents projets Arduino.  L'initiateur de cette idée est en effet l'instituteur Jean-Christophe CARRE dit "jc". Nos activités ont été réellement effectives à partir de Mars 2021. L'idée était d'être prêt pour les journées étudiantes qui avaient lieu le même mois. Notre groupe était organisé en cinq (5) membres sous la supervision de jc. Le squelette de l'imprimante 3D avait déjà été assemblé par jc. Le gros du travail résidait d'abord dans les branchements et les calculs liés au type de fonctionnement escompté de la machine. Il y avait aussi la configuration et le flashage de la firmware \textit{Marlin 2.0} (qui nous était totalement inconnue au début); Ceci pour permettre à la machine de fonctionner intelligemment et selon nos diverses attentes. Notre travail a porté également sur la conception 3D à partir du logiciel libre \textit{FreeCad} sous \textit{Linux}. Nous nous sommes également penché sur le dessin sur \textit{Inkscape} couplé avec la génération de \textit{g-code} pour les mouvements du Laser. Ce rapport rend compte de l'ensemble des activités menées sur ce projet. Il détaille aussi sur les branchements et configurations effectuées. Enfin il présente un guide pour aider à la manipulations de l'imprimante 3D.

\end{spacing}
\end{abstract}
