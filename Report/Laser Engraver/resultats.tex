\part{Travaux effectués et résultats}

\section{Partie 1 - Branchement et Mécanique}

Cette section n'est pas renseignée!
\section{Partie 2 - Code et configuration de la firmware}
À partir de la configuration par défaut, il faut dé-commenter le nécessaire et ajouter des paramètre fondamentales ou personnels. Toute section ou paramètre qui ne sera pas cité dans ce document est à sa valeur par défaut selon la configuration de marlin 2.0 .
Le code est subdivisé en plusieurs sections de paramètres.

\subsection{Hardware Info}
Il nous revenait dans cette section de : 
\begin{itemize}
	\item modifier la valeur du baudrate pour la fixer à 250000
	\item définir la motherboard convenable (BOARD\_RAMPS\_14\_EFF)
	\item désigner un nom pour l’imprimante (CUSTOM\_MACHINE\_NAME : rose)
\end{itemize}
Pour le deuxième tiret il revient à modifier le fichier \textbf{boards.h} pour choisir le paramétrage convenable.

\subsection{Extruder Info}
\begin{itemize}
	\item choix du nombre d'extrudeurs : 1
	\item diamètre de filament et nozzle\footnotemark (70 et singlenozzle)
\end{itemize}
\footnotetext{nozzle désigne la buse de l'extrudeur}

\subsection{Kinematics}
Le premier paramètre à configurer faisait référence à la cinématique de l'imprimante 3D. C'est à dire le style de mouvement de la tête de laser selon les axes XYZ. les instructions disaient de dé-commenter au moins une option de cinématique entre : COREXY, COREXZ, COREYZ, COREYX, COREZX, COREZY, MARKFORGED\_XY. Mais on se rend compte après que le style de mouvement de chacun de ces sept options ne correspond vraiment pas à nos attentes. Il se trouve qu'on peut laisser commenté toutes les options. Et il y aura un comportement par défaut qui correspondra au style de mouvement désiré. En effet, par défaut marlin fonctionnerait en cartésien (XYZ) (source \footnotemark)
\footnotetext{\href{https://www.lesimprimantes3d.fr/forum/topic/18036-probl\%C3\%A8me-pilotage-axe/?do=findComment&comment=230701}{Consulter le forum}}

\subsection{Endstops }
Dans cette rubrique qui concerne exclusivement les capteurs de fins de course (endstop), il fallait :
\begin{itemize}
	\item modifier la logique des endstops (true).
	\item fixer le stepper driver. Pour les pilotes non spécifiés, la valeur par défaut est A4988
	\item activer les pins (interrups) pour les capteurs de fin de course 
\end{itemize}

\subsection{Movement}
Cette section est propre aux paramètres de mouvements de l'imprimante.
\begin{itemize}
	\item spécification du microsteping de 1/4 considéré.
	\item spécification du rythme d'accélération pour ne pas faire souffrir la mécanique
\end{itemize}

\subsection{Drivers}
\begin{itemize}
	\item inversion du sens de rotation des moteurs selon les axes XYZ (false)
\end{itemize}

\subsection{Homing and Bounds}
\begin{itemize}
	\item modification de la taille du bed (X = 300, Y = 318)
\end{itemize}

\subsection{Additonal Features}
\begin{itemize}
	\item activation de l'estimation de temps d'impression : PRINTJOB\_TIMER\_AUTOSTART
\end{itemize}

\subsection{User Interface Language}
\begin{itemize}
	\item définition de la langue du moniteur LCD
	\item Activation du support de carte SD
	\item définition de la vitesse de transmission SPI (fixée à HALF\_SPEED)
\end{itemize}

\subsection{Encoder}
\begin{itemize}
	\item inversion du sens de la molette du moniteur
\end{itemize}

\subsection{LCD Controller}
\begin{itemize}
	\item configuration de l'écran (LCD controller with click-wheel.)
\end{itemize}

\newpage